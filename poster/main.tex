%%%%%%%%%%%%%%%%%%%%%%%%%%%%%%%%%%%%%%%%%
% Inspired from Jacobs Landscape Poster
% LaTeX Template
% Version 1.1 (14/06/14)
%
% Created by:
% Computational Physics and Biophysics Group, Jacobs University
% https://teamwork.jacobs-university.de:8443/confluence/display/CoPandBiG/LaTeX+Poster
% 
% Further modified by:
% Nathaniel Johnston (nathaniel@njohnston.ca)
%
% This template has been downloaded from:
% http://www.LaTeXTemplates.com
%
% License:
% CC BY-NC-SA 3.0 (http://creativecommons.org/licenses/by-nc-sa/3.0/)
%
%%%%%%%%%%%%%%%%%%%%%%%%%%%%%%%%%%%%%%%%%
% TODO: 
% [OK] Resumer les sections Introduction et Conclusion (via Copilot) 
% [OK] Faire le lien entre la porte et le sujet de la conférence - Transformation numérique - sécurité - potentiel d'être augmenté par l'IA.  
% [OK] Checker le rajout de deux institutions pour le poster.
% [OK] Corriger les legendes des images et tables 
% [OK] Revoir les références 
% [OK] Retrouver space pour l'image du de-vers (et déssiner les images) 
% [OK] Déssiner l'arbre de données et rôles (avant et actuel)
% [OK] Remplacer le titre AI et gouvernement par Perspectives et travaux futurs 
% [OK] Déplacer la photo de l'équipe dans rémerciements... 
% [OK] Déplacer l'image de l'app vers la 3e colonne - Image supprimée; substitué par l'image de cas d'utilisation, qui l'emploie.  
% Remplacer les images du code par un pseudo-code ou algorithme simplifié ... code? Image?
% Si possible, ajouter d'autres photos de la banque d'images de la porte (page d'attestation, application mobile, etc) pour être plus visuel que textuel. ? 
%%%%%%%%%%%%%%%%%%%%%%%%%%%%%%%%%%%%%%%%%

%----------------------------------------------------------------------------------------
%	PACKAGES AND OTHER DOCUMENT CONFIGURATIONS
%----------------------------------------------------------------------------------------

\documentclass[final]{beamer}

\usepackage[scale=1.24]{beamerposter} % Use the beamerposter package for laying out the poster

\usetheme{confposter} % Use the confposter theme supplied with this template

\setbeamercolor{block title}{fg=ngreen,bg=white} % Colors of the block titles
\setbeamercolor{block body}{fg=black,bg=white} % Colors of the body of blocks
\setbeamercolor{block alerted title}{fg=white,bg=dblue!70} % Colors of the highlighted block titles
\setbeamercolor{block alerted body}{fg=black,bg=dblue!10} % Colors of the body of highlighted blocks
% Many more colors are available for use in beamerthemeconfposter.sty

%-----------------------------------------------------------
% Define the column widths and overall poster size
% To set effective sepwid, onecolwid and twocolwid values, first choose how many columns you want and how much separation you want between columns
% In this template, the separation width chosen is 0.024 of the paper width and a 4-column layout
% onecolwid should therefore be (1-(# of columns+1)*sepwid)/# of columns e.g. (1-(4+1)*0.024)/4 = 0.22
% Set twocolwid to be (2*onecolwid)+sepwid = 0.464
% Set threecolwid to be (3*onecolwid)+2*sepwid = 0.708

\newlength{\sepwid}
\newlength{\onecolwid}
\newlength{\twocolwid}
\newlength{\threecolwid}
\setlength{\paperwidth}{48in} % A0 width: 46.8in
\setlength{\paperheight}{72in} % A0 height: 33.1in
\setlength{\sepwid}{0.024\paperwidth} % Separation width (white space) between columns
\setlength{\onecolwid}{0.22\paperwidth} % Width of one column
\setlength{\twocolwid}{0.464\paperwidth} % Width of two columns
\setlength{\threecolwid}{0.708\paperwidth} % Width of three columns
\setlength{\topmargin}{-0.5in} % Reduce the top margin size
%-----------------------------------------------------------

\usepackage{graphicx}  % Required for including images

\usepackage{booktabs} % Top and bottom rules for tables

\usepackage{adjustbox}

%----------------------------------------------------------------------------------------
%	TITLE SECTION 
%----------------------------------------------------------------------------------------

\title{The 4th Soldier: The Story of Brother Lt-Col Charles Alexander Young} % Poster title



% \title{\begin{adjustbox}{valign=t,flushleft}\includegraphics[width=1in]{LogoPort-E.png}\end{adjustbox} \\ Titre du document}

\author{ Julio Cesar Torres dos Santos} % Author(s)

\institute{Loge Albion, No.2 - Grand Loge du Québec} % Institution(s)

%----------------------------------------------------------------------------------------

\begin{document}
 
\addtobeamertemplate{block end}{}{\vspace*{2ex}} % White space under blocks
\addtobeamertemplate{block alerted end}{}{\vspace*{2ex}} % White space under highlighted (alert) blocks

\setlength{\belowcaptionskip}{2ex} % White space under figures
\setlength\belowdisplayshortskip{2ex} % White space under equations

\begin{frame}[t] % The whole poster is enclosed in one beamer frame

\begin{columns}[t] % The whole poster consists of three major columns, the second of which is split into two columns twice - the [t] option aligns each column's content to the top

\begin{column}{\sepwid}\end{column} % Empty spacer column

\begin{column}{\onecolwid} % The first column

%----------------------------------------------------------------------------------------
%	OBJECTIVES
%----------------------------------------------------------------------------------------

\begin{alertblock}{Objectives}

Lt-Col Charles Alexander Young was a military and freemasonry member... 

\end{alertblock}

%----------------------------------------------------------------------------------------
%	INTRODUCTION
%----------------------------------------------------------------------------------------

\begin{block}{Introduction}

Introduction to the presentation...

\end{block}


%----------------------------------------------------------------------------------------
%	MÉTHODOLOGIE - METHODS
%----------------------------------------------------------------------------------------

\begin{block}{Methodologie}

This research was made doing... 

\begin{figure}
\includegraphics[width=1.0\linewidth]{Port-E_data01.png}
\caption{ L'attestation vérifiable auto-déclarée d'identité numérique} 
\end{figure}

\end{block}

\end{column} % End of the first column


\begin{column}{\sepwid}\end{column} % Empty spacer column

\begin{column}{\twocolwid} % Begin a column which is two columns wide (column 2)

%----------------------------------------------------------------------------------------
%	CAS D'UTILISATION - USE CASE
%----------------------------------------------------------------------------------------

\begin{alertblock}{Bro Lt-Col. Charles Alexander Young}

Brother Charles Alexander ...

\end{alertblock} 

\begin{figure}
\includegraphics[width=0.9\linewidth]{Port-E-LDAP.png}
\caption{ Modèle de données d'accès}
\end{figure}

%----------------------------------------------------------------------------------------

\begin{columns}[t,totalwidth=\twocolwid] % Split up the two columns wide column again

\begin{column}{\onecolwid} % The first column within column 2 (column 2.1)


%----------------------------------------------------------------------------------------
%	RÉSULTATS - RESULTS
%----------------------------------------------------------------------------------------

\begin{block}{Résultats}

Results of the research ...
\end{block}



%----------------------------------------------------------------------------------------

\end{column} % End of column 2.1


\begin{column}{\onecolwid} % The second column within column 2 (column 2.2)

%----------------------------------------------------------------------------------------
%	PERSPECTIVES ET TRAVAUX FUTURS - PROSPECTS AND FURTHER WORK 
%----------------------------------------------------------------------------------------

%\setbeamercolor{block alerted title}{fg=black,bg=norange} % Change the alert block title colors
%\setbeamercolor{block alerted body}{fg=black,bg=white} % Change the alert block body colors

\begin{alertblock}{Perspectives and further Work}

Further works...

\end{alertblock}

%----------------------------------------------------------------------------------------

\end{column} % End of column 2.2

\end{columns} % End of the split of column 2

\begin{column}{\twocolwid} % Begin a column which is two columns wide (column 2)

%----------------------------------------------------------------------------------------
%	IMAGES
%----------------------------------------------------------------------------------------

\begin{figure}
\includegraphics[width=0.8\linewidth]{Collage_02.png}
\caption{Certains visuels sur l'expérimentation}
\end{figure}

%----------------------------------------------------------------------------------------

\end{column} % End of last two column

\end{column} % End of the second column


\begin{column}{\sepwid}\end{column} % Empty spacer column

\begin{column}{\onecolwid} % The third column

%----------------------------------------------------------------------------------------
%	IMAGE ATTESTATION NUMÉRIQUE
%----------------------------------------------------------------------------------------

%\begin{figure}
%\includegraphics[width=0.6\linewidth]{attestation_portefeuille.png}
%\caption{Application mobile de portefeuille numérique}
%\end{figure}

%----------------------------------------------------------------------------------------
%	CONCLUSION
%----------------------------------------------------------------------------------------

\begin{block}{Conclusion}

Conclusions

\end{block}

%----------------------------------------------------------------------------------------
%	INFORMATION ADDITIONELLE - ADDITIONAL INFORMATION
%----------------------------------------------------------------------------------------

\begin{block}{Additional Information}
Additional Info...
\end{block}

%----------------------------------------------------------------------------------------
%	RÉFÉRENCES - REFERENCES
%----------------------------------------------------------------------------------------

\begin{block}{Références}

\nocite{*} % Insert publications even if they are not cited in the poster
\small{\bibliographystyle{unsrt}
\bibliography{RefBiblio}\vspace{0.75in}}

\end{block}

%----------------------------------------------------------------------------------------
%	REMERCIEMENTS - ACKNOWLEDGEMENTS
%----------------------------------------------------------------------------------------

\setbeamercolor{block title}{fg=red,bg=white} % Change the block title color

\begin{block}{Acknowledgements}

\small{\rmfamily{We would like to thank Bro Charles Alexander's granddaughter, Jayne Thompson-Castagnette, who has kindly supplied informations, pictures, and her time for the research of her family's members.}} \\

\end{block}

%----------------------------------------------------------------------------------------
%	COORDONNÉES - CONTACT INFORMATION
%----------------------------------------------------------------------------------------

\setbeamercolor{block alerted title}{fg=black,bg=norange} % Change the alert block title colors
\setbeamercolor{block alerted body}{fg=black,bg=white} % Change the alert block body colors

\begin{alertblock}{Coordonnées}

\begin{itemize}
% \item Web: \href{https://www.quebec.ca/gouvernement/ministere/cybersecurite-numerique}{https://www.quebec.ca/gouvernement/ministere/cybersecurite-numerique}
% \item GitHub: \href{https://github.com/CQEN-QDCE}{https://github.com/CQEN-QDCE}
\item Email: \href{mailto:cqen@mcn.gouv.qc.ca}{juliozohar@gmail.com}
\end{itemize}

\end{alertblock}

% Ajouter de l'espace afin de repousser le logo du MCN complètement en bas du poster
\vspace{15cm}

\begin{center}
\begin{tabular}{ccc}
% \includegraphics[width=1.0\linewidth]{mcn.png}
\end{tabular}
\end{center}

%----------------------------------------------------------------------------------------

\end{column} % End of the third column

\end{columns} % End of all the columns in the poster

\end{frame} % End of the enclosing frame

\end{document}
